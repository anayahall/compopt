\documentclass[12 pt,]{article}
\usepackage{lmodern}
\usepackage{amssymb,amsmath}
\usepackage{ifxetex,ifluatex}
\usepackage{fixltx2e} % provides \textsubscript
\ifnum 0\ifxetex 1\fi\ifluatex 1\fi=0 % if pdftex
  \usepackage[T1]{fontenc}
  \usepackage[utf8]{inputenc}
\else % if luatex or xelatex
  \ifxetex
    \usepackage{mathspec}
  \else
    \usepackage{fontspec}
  \fi
  \defaultfontfeatures{Ligatures=TeX,Scale=MatchLowercase}
\fi
% use upquote if available, for straight quotes in verbatim environments
\IfFileExists{upquote.sty}{\usepackage{upquote}}{}
% use microtype if available
\IfFileExists{microtype.sty}{%
\usepackage{microtype}
\UseMicrotypeSet[protrusion]{basicmath} % disable protrusion for tt fonts
}{}
\usepackage[margin=.75in]{geometry}
\usepackage{hyperref}
\hypersetup{unicode=true,
            pdftitle={compost distribution},
            pdfauthor={Anaya Hall},
            pdfborder={0 0 0},
            breaklinks=true}
\urlstyle{same}  % don't use monospace font for urls
\usepackage{graphicx,grffile}
\makeatletter
\def\maxwidth{\ifdim\Gin@nat@width>\linewidth\linewidth\else\Gin@nat@width\fi}
\def\maxheight{\ifdim\Gin@nat@height>\textheight\textheight\else\Gin@nat@height\fi}
\makeatother
% Scale images if necessary, so that they will not overflow the page
% margins by default, and it is still possible to overwrite the defaults
% using explicit options in \includegraphics[width, height, ...]{}
\setkeys{Gin}{width=\maxwidth,height=\maxheight,keepaspectratio}
\IfFileExists{parskip.sty}{%
\usepackage{parskip}
}{% else
\setlength{\parindent}{0pt}
\setlength{\parskip}{6pt plus 2pt minus 1pt}
}
\setlength{\emergencystretch}{3em}  % prevent overfull lines
\providecommand{\tightlist}{%
  \setlength{\itemsep}{0pt}\setlength{\parskip}{0pt}}
\setcounter{secnumdepth}{0}
% Redefines (sub)paragraphs to behave more like sections
\ifx\paragraph\undefined\else
\let\oldparagraph\paragraph
\renewcommand{\paragraph}[1]{\oldparagraph{#1}\mbox{}}
\fi
\ifx\subparagraph\undefined\else
\let\oldsubparagraph\subparagraph
\renewcommand{\subparagraph}[1]{\oldsubparagraph{#1}\mbox{}}
\fi

%%% Use protect on footnotes to avoid problems with footnotes in titles
\let\rmarkdownfootnote\footnote%
\def\footnote{\protect\rmarkdownfootnote}

%%% Change title format to be more compact
\usepackage{titling}

% Create subtitle command for use in maketitle
\newcommand{\subtitle}[1]{
  \posttitle{
    \begin{center}\large#1\end{center}
    }
}

\setlength{\droptitle}{-2em}
  \title{compost distribution}
  \pretitle{\vspace{\droptitle}\centering\huge}
  \posttitle{\par}
  \author{Anaya Hall}
  \preauthor{\centering\large\emph}
  \postauthor{\par}
  \date{}
  \predate{}\postdate{}


\begin{document}
\maketitle

\(i\) : Index of county (\(1,...,n\))\\
\(j\) : Index of facilities (\(1,...,m\))\\
\(k\) : Index on rangelands (\(1,...,p\))

\[GWP = 
\underbrace{\sum_{i=1}^{n} \sum_{j=1}^{m} h \cdot s_{ij} \cdot D_{ij} }_{\text{transport from county to facility}} + 
\underbrace{\sum_{i=1}^{n} f \cdot (1-\sum_{j=1}^{m}s_{ij}) }_{\text{waste that remains in county}} +  
\underbrace{\sum_{i=1}^{n} \sum_{j=1}^{m} t \cdot s_{ij}}_{\text{processing emissions}} + 
\underbrace{\sum_{k=1}^{p} \sum_{j=1}^{m} h \cdot L_{jk}d_{jk} }_{\text{transport from facility to land}} +  
\underbrace{\sum_{k=1}^p p \cdot d_{jk}}_{\text{spreading compost}} \]

\[ Cost = 
\underbrace{\sum_{i=1}^{n} \sum_{j=1}^{m} d \cdot D_{ij} s_{ij} }_{\text{transport from county to facility}} +  
\underbrace{\sum_{i=1}^{n} \sum_{j=1}^{m} e \cdot L_{jk}d_{jk} }_{\text{transport from facility to land}} +  
\underbrace{\sum_{i=1}^n k \cdot A_k}_{\text{cost to spread compost}} \]

Intake for each facility is sum of the proportion taken in from \(c_i\)
for \(i = 1,...,n\)\\
\[I_j = \sum_{i=1}^{n} s_{ij}\]

Output of each facility is equal to intake converted into compost\\
\[O_j = c \cdot I_j\]

Total compost applied in each county is the sum of the proporion of
output from \(f_j\) for \(j = 1,...,m\)\\
\[A_k = \sum_{j=1}^{m} d_{jk}\]

subject to:

\(I_j \leq F_j\)\\
\(A_i \leq C_i\)\\
\(\sum_{j=1}^{m} s_{ij} \leq W_i\)\\
\(\sum_{i=1}^{n} d_{jk} \leq F_j\)

\$0 \leq s\_\{ij\} \$\\
\$0 \leq d\_\{jk\} \$

where

\(s_{ij}\) = Quantity of \(W_i\) to send to \(f_j\)\\
 \(d_{jk}\) = Quantity of facility \(f_j\) output to send to \(c_i\)
working land\\
 \(D_{ij}\) = Distance to haul to facility from county centroid\\
\(L_{jk}\) = Distance from facility to working land\\
\(W_i\) = Waste available in county \(i\)\\
\(F_j\) = Intake capacity of facility \(j\)\\
\(C_i\) = Amount of output county \(i\) can take in (based on amount of
land)

and

\(S\) = Sequestration rate\\
\(c\) = Conversion factor of waste into compost (\%)\\
\(f\) = Emission factor for waste left in county
(\(\frac {CO2_{e}}{ton}\))\\
\(g\) = Emission factor for compost stranded at facility
(\(\frac {CO2_{e}}{ton}\))\\
\(h\) = Transportation emission factor
(\(\frac {CO2_{e}}{ton \cdot km}\)) (separate??)\\
\(p\) = Emission factor for compost production
(\(\frac {CO2_{e}}{ton}\))\\
\(e\) = Cost to haul away from facility to land
(\(\frac {\$}{ton \cdot km}\))\\
\(d\) = Cost to haul to facility from county
(\(\frac {\$}{ton \cdot km}\))\\
\(k\) = Cost to apply compost to fields (\(\frac {\$}{ton}\))


\end{document}
